\documentclass[11pt]{article}
\usepackage[parfill]{parskip} %to make indents on returns go away
\usepackage{graphicx} %to allow graphics
\usepackage{amsmath} %to allow most math
\usepackage[letterpaper, total={5in, 7in}]{geometry}
\usepackage{tikz} 
\usepackage{verbatim}

%making new names for common things
\newcommand{\beq}{\begin{equation}}
\newcommand{\eeq}{\end{equation}}
\newcommand{\beqa}{\begin{equation}\begin{aligned}}
\newcommand{\eeqa}{\end{aligned}\end{equation}}
\newcommand{\Dt}{\Delta t}

%for the table of contents
\usepackage{hyperref}
\hypersetup{
    colorlinks,
    citecolor=magenta,
    filecolor=black,
    linkcolor=black,
    urlcolor=blue
}

%for Tikz
\usetikzlibrary{arrows,shapes,positioning}
\usetikzlibrary{decorations.markings}
\tikzstyle arrowstyle=[scale=1]
\tikzstyle directed=[postaction={decorate,decoration={markings,
    mark=at position .65 with {\arrow[arrowstyle]{stealth}}}}]
\tikzstyle reverse directed=[postaction={decorate,decoration={markings,
    mark=at position .65 with {\arrowreversed[arrowstyle]{stealth};}}}]


\title{Template in TeX}
\author{dbr}

\begin{document}
\maketitle
\tableofcontents

\section{This is a big heading}

Notice I made a table of contents above, this can be useful. 
It can also be useful to write comments to yourself in the source code, use the $\%$ sign to do that. 
That means the percent sign is special and needs to be written special in TeX, that's a common error.

\subsection{And this is a sub heading}

You can make a list:

\begin{itemize}
\item first let's learn some easy math stuff
\end{itemize}


Maybe you have the most brilliant simple equation

\beq
E=mc^2
\eeq

or maybe you need 2 lines

\beqa
\frac{\partial x}{\partial t}&=\dot{x}=\frac{\Delta x}{\Dt} = v\\
v&=\sqrt{2E/m}
\eeqa

maybe it's not a super important equation, and thus doesn't need to be numbered

\[ F=ma \]

and if you want a matrix, like an $m\times n $ matrix $\mathcal{M}$?

\[
\mathcal{M} = \begin{bmatrix}
    x_{11} & x_{12} & x_{13} & \dots  & x_{1n} \\
    x_{21} & x_{22} & x_{23} & \dots  & x_{2n} \\
    \vdots & \vdots & \vdots & \ddots & \vdots \\
    x_{m1} & x_{m2} & x_{m3} & \dots  & x_{mn}
\end{bmatrix}
\]

If you want to denote code, typically you might write: I used \texttt{ode15} in Matlab.

I like to make {\bf vectors} bold, e.g. $\mathbf{v}=a\hat{x}+b\hat{y}$, and sometimes greek letters need to be bolded as well $\boldsymbol{\xi}$

\subsection{Bibliography stuff}

I often use the hyperref package so that I can color my citations and make them clickable, for example Ref.~\cite{Sompayrac2011}, has been essential to my immunology learning. Note the tilde in the source document it makes the spacing correct.

Using .bib files is really helpful for organisation. It's possible to copy these citations straight from google scholar.

\section{Going nuts with Tikz}

You can go crazy making pictures in TeX too, and reference them like this: in Fig.~\ref{fig1}, we see someone elses hard work. 


\begin{figure}[h!]\centering
%\includegraphics[width=3.2in]{PIC_hivLatency} %use to add your own

\begin{tikzpicture}

    % define coordinates
    \coordinate (O) at (0,0) ;
    \coordinate (A) at (0,4) ;
    \coordinate (B) at (0,-4) ;
    
    % media
    \fill[blue!25!,opacity=.3] (-4,0) rectangle (4,4);
    \fill[blue!60!,opacity=.3] (-4,0) rectangle (4,-4);
    \node[right] at (2,2) {Air};
    \node[left] at (-2,-2) {Water};

    % axis
    \draw[dash pattern=on5pt off3pt] (A) -- (B) ;

    % rays
    \draw[red,ultra thick,reverse directed] (O) -- (130:5.2);
    \draw[blue,directed,ultra thick] (O) -- (-70:4.24);

    % angles
    \draw (0,1) arc (90:130:1);
    \draw (0,-1.4) arc (270:290:1.4) ;
    \node[] at (280:1.8)  {$\theta_{2}$};
    \node[] at (110:1.4)  {$\theta_{1}$};
\end{tikzpicture}
\caption{This is a tikz figure I stole from the internet. Lol.}
\label{fig1}
\end{figure}

\bibliographystyle{unsrt}
\begin{thebibliography}{10}
\bibitem{Sompayrac2011} Sompayrac, Lauren M. How the immune system works. John Wiley \& Sons, 2011.
\end{thebibliography}


\end{document}  
